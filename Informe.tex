\documentclass[10pt]{extarticle}

%Paquetes utilizados en esta tarea
\usepackage{fullpage}
\usepackage[utf8]{inputenc}
\usepackage[spanish]{babel}
\usepackage{epsfig}
\usepackage{amsmath}
\usepackage{amssymb}
\usepackage{epstopdf}
\usepackage[hidelinks]{hyperref}
\usepackage{xcolor}
\usepackage{algorithmic}
\usepackage[nothing]{algorithm}

%Definiciones de comandos, para reutilizar secuencias frecuentes o ahorrar
%código
\newcommand{\RR}{\mathbb{R}}
\newcommand{\lb}{\\~\\}
\newcommand{\la}{\leftarrow}

\newcommand{\twopartdef}[4]
{
	\left\{
		\begin{array}{ll}
			#1 &  \text{si }#2 \\
			#3 &  \text{si }#4
		\end{array}
	\right.
}

\newcommand{\threepartdef}[6]
{
	\left\{
		\begin{array}{ll}
			#1 &  \text{si }#2 \\
			#3 &  \text{si }#4 \\
			#5 &  \text{si }#6
		\end{array}
	\right.
}

\makeatletter

\makeatother

\begin{document}

\begin{tabular}{ccl}
 \begin{tabular}{c}
 \includegraphics[width=2.5cm]{imgs/logo.pdf}
\end{tabular}
&\ \ \ &
\begin{tabular}{l}
 PONTIFICIA UNIVERSIDAD CATÓLICA DE CHILE               \\
 DEPARTAMENTO DE CIENCIA DE LA COMPUTACIÓN              \\
 IIC3524 {-} Tópicos avanzados de sistemas distribuidos \\
\end{tabular}
\end{tabular}

\begin{center}
 \bf {\Huge Tarea 1}

 \vspace{0.2cm}
 \bf 16 de mayo de 2017

 \vspace{0.2cm}
 \bf Nicolás Gebauer {-} 13634941

 \vspace{0.2cm}
 \bf \href{https://github.com/negebauer}{\color{blue!60} @negebauer} {-} \href{https://github.com/negebauer/IIC3524-T1}{\color{blue!60}repo T1}
 \noindent\rule{16cm}{0.05pt}

 \vspace{0.5cm}
 % \bf {\huge Análisis}
\end{center}

\subsection*{Pregunta 1}
Esta es la respuesta de la pregunta 1

\end{document}
